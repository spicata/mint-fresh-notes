% Options for packages loaded elsewhere
\PassOptionsToPackage{unicode}{hyperref}
\PassOptionsToPackage{hyphens}{url}
%
\documentclass[
]{article}
\usepackage{amsmath,amssymb}
\usepackage{lmodern}
\usepackage{iftex}
\ifPDFTeX
  \usepackage[T1]{fontenc}
  \usepackage[utf8]{inputenc}
  \usepackage{textcomp} % provide euro and other symbols
\else % if luatex or xetex
  \usepackage{unicode-math}
  \defaultfontfeatures{Scale=MatchLowercase}
  \defaultfontfeatures[\rmfamily]{Ligatures=TeX,Scale=1}
\fi
% Use upquote if available, for straight quotes in verbatim environments
\IfFileExists{upquote.sty}{\usepackage{upquote}}{}
\IfFileExists{microtype.sty}{% use microtype if available
  \usepackage[]{microtype}
  \UseMicrotypeSet[protrusion]{basicmath} % disable protrusion for tt fonts
}{}
\makeatletter
\@ifundefined{KOMAClassName}{% if non-KOMA class
  \IfFileExists{parskip.sty}{%
    \usepackage{parskip}
  }{% else
    \setlength{\parindent}{0pt}
    \setlength{\parskip}{6pt plus 2pt minus 1pt}}
}{% if KOMA class
  \KOMAoptions{parskip=half}}
\makeatother
\usepackage{xcolor}
\usepackage{longtable,booktabs,array}
\usepackage{calc} % for calculating minipage widths
% Correct order of tables after \paragraph or \subparagraph
\usepackage{etoolbox}
\makeatletter
\patchcmd\longtable{\par}{\if@noskipsec\mbox{}\fi\par}{}{}
\makeatother
% Allow footnotes in longtable head/foot
\IfFileExists{footnotehyper.sty}{\usepackage{footnotehyper}}{\usepackage{footnote}}
\makesavenoteenv{longtable}
\setlength{\emergencystretch}{3em} % prevent overfull lines
\providecommand{\tightlist}{%
  \setlength{\itemsep}{0pt}\setlength{\parskip}{0pt}}
\setcounter{secnumdepth}{-\maxdimen} % remove section numbering
\ifLuaTeX
  \usepackage{selnolig}  % disable illegal ligatures
\fi
\IfFileExists{bookmark.sty}{\usepackage{bookmark}}{\usepackage{hyperref}}
\IfFileExists{xurl.sty}{\usepackage{xurl}}{} % add URL line breaks if available
\urlstyle{same} % disable monospaced font for URLs
\hypersetup{
  pdftitle={4,2a - Beta Decay (P)},
  hidelinks,
  pdfcreator={LaTeX via pandoc}}

\title{4,2a - Beta Decay (P)}
\author{}
\date{}

\begin{document}
\maketitle

\hypertarget{a---beta-decay}{%
\section{4,2a - Beta Decay}\label{a---beta-decay}}

Beta decay is another way that unstable nuclei can achieve stability
(4,2a,a - Beta or Alpha decay). It occurs when neutrons transform into a
proton (positron) and a electron (negatron) (4,2b - Beta Decay Unlimited
(P)), increasing stability by increasing the proton to neutron ratio.
The electron is tiny, and so it can be ejected from the nucleus as an
incredibly high speed\textsuperscript{{[}1{]}}. This is the {\(\beta\)}
particle.

Beta decay has higher penetration, as the smaller particle allows it to
fit through more gaps in a surface than the {\(\alpha\)} particle.
However, it has a lower ionisation ability as it needs to not only be
closer due its size\textsuperscript{{[}2{]}}, but also because it has a
weaker charge.\textsuperscript{{[}3{]}}

\begin{longtable}[]{@{}ll@{}}
\toprule()
Type of Decay & General Formula (A is atomic Mass, Z is atomic
number) \\
\midrule()
\endhead
Alpha Decay &
{\(_{Z}^{A}X \rightarrow_{Z + 1}^{A}Y +_{- 1}^{0}\beta\)} \\
\bottomrule()
\end{longtable}

It often appears with gamma radiation (4,2a,2a - Gamma Decay). A prime
example of beta decaying substance would be bismuth-214.

« 4 - Alpha Decay : 4,2b - Beta Decay Unlimited (P) »

\begin{center}\rule{0.5\linewidth}{0.5pt}\end{center}

\begin{enumerate}
\tightlist
\item
  \protect\hypertarget{fn-1-4ab4ebdaf9a4c5a0}{}{As the amount of protons
  in a nuclei increases, does the ``ejection speed'' also increase?}
\item
  \protect\hypertarget{fn-2-4ab4ebdaf9a4c5a0}{}{Why does size play a
  part in ionisation}
\item
  \protect\hypertarget{fn-3-4ab4ebdaf9a4c5a0}{}{Because the {\(- 1\)}
  charge of {\(\beta\)} particle is weaker than the {\(+ 2\)} charge of
  the {\(\alpha\)} particle.}
\end{enumerate}

\end{document}
